\documentclass{beamer}
\usepackage[utf8]{inputenc}
\usepackage{xcolor}
\usepackage{listings}
\usepackage{graphicx}
\usetheme{Warsaw}

\title{Vidéo surveillance, Streaming vidéo et contrôle de caméra via Android }
\author{Jérôme NAHELOU, Quentin NEBOUT, Romain SOLVE, Fabien QUINTARD}
\institute{\large{Chargé de Projet : David BROMBERG}\\ \bigskip{}
\small{Université Bordeaux 1}}
\date{29 mars 2010}

\begin{document}
\frame[plain]{\titlepage}

\AtBeginSection[]{
\begin{frame}<beamer>
\frametitle{Plan}
\tableofcontents[currentsection]
\end{frame}}

\begin{frame}
\frametitle{Plan de l'exposé}
\tableofcontents
\end{frame}

\section{Introduction}
  \begin{frame}
   \frametitle{Description}
  Introduction
   \begin{itemize}
    \item<2-> Android: OS pour appareil mobile, basé sur noyaux linux
    \item<3-> Choix: M-JPEG, HTTP-GET
   \end{itemize}
  \end{frame}

\section{Aspect général de l'application}
  \subsection{Description}
  \begin{frame}
   \frametitle{Description}
  Aspect général de l'application convivial
   \begin{itemize}
    \item<2-> Astuces pour mieux connaître les fonctionnalités
    \item<3-> Application Multi-langue avec détection automatique
   \end{itemize}
  \end{frame}

\section{Simple vue}
\subsection{Spécifications}
 \begin{frame}
   \frametitle{Spécifications}

\end{frame}

\section{Multi-vue}
\subsection{Spécifications}
 \begin{frame}
   \frametitle{Spécifications}

\end{frame}

\section{Contrôle de la caméra}
\subsection{Commandes}
 \begin{frame}
   \frametitle{Commandes}

\end{frame}

\section{Détection de mouvements}
\subsection{Comment ca fonctionne}
 \begin{frame}
   \frametitle{Comment ca fonctionne}

\end{frame}

\section{Tests Unitaires}
\subsection{Principe}
 \begin{frame}
   \frametitle{Principe}
Le but des tests unitaires est de:
\begin{itemize}
    \item<2-> Garantir le bon fonctionnement de l'application
    \item<3-> Verifier si les fonctions sortent bien
    \item<4-> 
   \end{itemize}
\end{frame}

\section{Conclusion}
\subsection{Optimisations}
 \begin{frame}
   \frametitle{Optimisations}

\end{frame}

\end{document}