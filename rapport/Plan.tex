- Introduction
  - Introduction au domaine
  - Introduction au projet

- Etude de l'existant
  - Plateforme Android (version 2.0-2.1)
  - Protocoles de transfert
    ** intro sur la cam **
    * HTTP
      - GET
       * def
       * exemple
      - POST
    * RTSP
      - OPTIONS
      - DESCRIBE
      - SETUP
      - PLAY
      - PAUSE
      - RECORD
      - TEARDOWN
     ** explication des choix **
  - Formats de vidéo
     * MPEG-4
      - def
      - avantages
      - inconveniants
     * MJPEG
      - def
      - avantages
      - inconveniants
      - justification
      - exemple de paquet mjpeg sur http      
  - Format son
     * G.711 μ-law
     * G.726
     * AAC
     
  

- Besoins fonctionnels / non fonctionnels


- Implémentation :

  - Type caméra 
    * Objet contenant les champs toto titi tutu + descriptif (utilitée)
    * Construction de l'objet via AddCam / Modification via EditCam
    * L'url peut etre construit via QrCode (utilisation lib xzing si non existe
    lien vers market)
    
  - Communication avec la caméra
    * Envoie de requete http GET -> sendCommand + utilisation d'un encodeur
      64bits pour chiffrer le mdp
    * Requete GET http pour recuperer des infos (loadConfig) + analyse
    * Liste des commandes utilisees pour diriger/recuperer les infos de la cam 
    
  - Gestion du flux vidéo (simple / multi-vue, Video)
    * 2 classes : 
        - Simple vue :
            * ConnectivityManager (wifi / reseau mobil) pour fixer la
              resolution de la video 
            * Utilisation Mjpeg view pour ameliorer les perfs
            * Mise a dispo de l'interface de controle camera expliquer ci
              dessous (menu, utilisation du LayoutInflater pour
              superposer/supprimer les views)
         - Multi vue : 
            * Mpjeg view personnalisé pour limiter les fps
               - handler 
               - schema exemple maj UI
            * Layout personnalisé
            * 1er onTouch : choix + demarrage d'une cam
            * 2nd onTouch : arrete de la cam
            * onLongTouch -> envoie vers simple vue pour personnaliser
    * wakelock pour empecher la mise en veille sur les 2 classes
    
  - Contrôle de la caméra (CameraControl, TouchListener)
    * fonctions dispo si pas dispo bouton desactivé
    * onTouch : 
        -- Recuperer les points avec : 
         - ACTION_DOWN
         - ACTION_POINTER_DOWN
         - ACTION_UP
         - ACTION_POINTER_UP
        -- Regler la sensibilité pour faire la correspondance avec les valeures
           utilisé par la camera
        -- Fonction de geometries   
    
  - Fonctionnalités avancés
    * Snapshot
      - Pop up choix de la resolution
      - Sauvegarde dans "/sdcard/com.myapps.camera/"
      - Format jpeg
      - Notification de la capture avec lancement : intent gallery par defaut
  - Détection de mouvement
    * Présentation (explication du mécanisme) 
    * utilisation de service Android
      - Demarre au lancement de l'application
      - Envoie de tache lors de l'activation (handler)
      -- Gestionnaire de notification en cours
       - Lancement de la camera + groupe pour desactiver en cas de retour
       
  - Paramètres / préférences (import / export) / Shared Preferences
    -- Import/Export en xml via xmlIO
     - Popup chemin destination
     - Pas de save des login/pass par securitée
    -- SharedPreferences pour regler les TO, sensibilité, delai

  - Multilingue & Interfacec & Partage
    - fr - en - de - es
    - Astuces au demarrage
    - Utilisation de style pour definir les couleurs, background, police, taille
      du texte
    - Partager l'app (message personalisé) par bluetoth, mail, facebook, gmail,
      messages
      

- Tests
  - Utilisation de plusieurs telephones (
    * Motorola millestone (OS : 2.0)
    * Samsung Galaxy S (I9000 OS : 2.2.1 v.I9000XXJPX) 
    * Samsung Galaxy (I7500 OS : 2.2)
  - Performance  réseau (WIFI / 3G) - Réponse serveur (contrôles)
  - Fiabilité de détection

- Optimisations / Extensions futures
  - Enregistrement vidéo / Retouche image
  - Edition des fenêtres de détection  de mouvement / Evénements suite à
    la détection (mail / snapshot)
  - Compte utilisateur (admin / viewer) => gestion paramètres caméra à distance

- Conclusion