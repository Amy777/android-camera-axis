\chapter{Etude de l'existant}
\section{Le système d'exploitation Android}La première version d'Android (1.0)
est parue en septembre 2008, développée par Google et l'Open Handset Alliance.
De nombreuses versions depuis sont sorties pour mettre à jour l'OS,
permettant ainsi de corriger les bugs et d'ajouter des fonctionnalités. Voici les dernières versions disponibles:
\newline
\begin{itemize}
  \item[o] \textbf{1.6 (Donut)} :
  Version basée sur le noyaux linux 2.6.29, sortie en septembre 2009.
  \begin{itemize}
    \item Interface native pour l'appareil photo, la caméra et la galerie
    \item Possibilité de sélectionner plusieurs photos pour la suppression
    \item Mise à jour de la détection vocale
    \item Possibilité de recherche dans les favoris, l'historique, les
    contacts et Internet
    \item Mise à jour de la technologie CDMA/EVDO, 802.1x, VPNs
    \item Support de la résolution WVGA
    \newline
  \end{itemize}
  
  
  \item[o] \textbf{2.0/2.1 (Eclair)} :
  Version basée sur le noyaux linux 2.6.29, sortie en décembre 2009 / janvier
  2010.
  \begin{itemize}
    \item Réorganisation de l'UI
    \item Support de nouvelles résolutions d'écran
    \item Support du HTML 5
    \item Meilleur contraste pour les fonds d'écran
    \item Amélioration de vitesse Hardware
    \item Classe MotionEvent améliorée pour supporter les multiples ``touch
    event''
    \item Zoom digital
    \item Bluetouth 2.1
    \item Microsoft Exchange Server
    \newline
  \end{itemize}
  
  \item[o] \textbf{2.2 (Froyo)} :
  Version basée sur le noyaux linux 2.6.32, sortie en mai 2010.
  \begin{itemize}
    \item Optimisations générales de la vitesse, de la mémoire et des performances
    d'Android OS.
    \item Intégration du moteur JavaScript V8 de Chrome dans le navigateur
    \item Hotspot Wi-fi et USB Thetering
    \item Support des mots de passe numériques et alphanumériques
    \item Support de l'upload de fichiers dans le navigateur
    \item Support de l'installation d'applications sur la mémoire extensible
    \item Support des écrans à haute densité de pixels (320 dpi)
    \item Support d'Adobe Flash 10.1
    \newline
  \end{itemize}
  
  \item[o] \textbf{2.3 (Gingerbread)} :
  Version basée sur le noyaux linux 2.6.35, sortie en décembre 2010.
  \begin{itemize}
    \item Support des grands écrans à résolutions extra-larges (WXGA et plus)
    \item Support de la VoIP et SIP
    \item Support des formats vidéo WebM/VP8, et de l'encodage audio AAC
    \item Nouveaux effets audio tels que la réverbération, l'égalisation, la
    virtualisation du casque audio et l'accentuation des graves
    \item Support du NFC
    \item Amélioration de la fonction copier/coller et sélection du texte
    \item Refonte du clavier virtuel (multi-touch) et de l'autocomplétion
    \item Garbage collector pour de meilleures performances
    \item Support de nouveaux capteurs (comme le gyroscope et le baromètre)
    \item Ajout d'un gestionnaire de téléchargement
    \item Amélioration de la gestion de l'alimentation et du contrôle des
    applications
    \item Passage au système de fichiers ext4
    \newline
    \end{itemize}
\end{itemize}