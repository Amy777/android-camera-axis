	\section{Les protocoles de transfert}
		\subsection{HTTP}
		Le protocole \textit{HTTP} (HyperText Transfer Protocol) permet d'intéragir
		aisément avec la caméra et donne la possiblité d'utiliser le mécanisme PTZ de la
		caméra (Pan Tilt Zoom), de faire une capture d'écran ou encore d'activer la
		détection de mouvements, tout cela par de simples requêtes. En effet, la
		méthode GET transmet les données via l'URL, ce qui permet d'utiliser toutes les fonctionnalités de la caméra par simple appels à des
		adresses définies. Exemple d'URL:
		\begin{lstlisting}
		http://<nom-serveur>/axis-cgi/jpg/image.cgi?resolution=320*240&camera=1
		\end{lstlisting}
		Permet de récuperer une capture au format JPG, de résolution
		320*240 pixels.
		\begin{lstlisting}
		http://<nom-serveur>/axis-cgi/com/ptz.cgi?rpan=15&rtilt=25
		\end{lstlisting}
		Fait bouger la caméra de 15 unités vers la droite et de 25
		unités vers le haut.
		
		Si la requête nécessite le retour de données, celles-ci seront contenues
		dans le corps de la réponse. Par exemple, dans le cas où nous envoyons la
		requête pour récupérer une capture (décrite ci-dessus). L'image sera
		contenue dans la reponse sous la forme : 
		\begin{lstlisting}
HTTP/1.0 200 OK\r\n
Content-Type: image/jpeg\r\n
Content-Length: 15656\r\n
\r\n
<JPEG image data>\r\n
\end{lstlisting}
		\subsection{RTSP}
		Le protocole \textit{RTSP} (Real Time Streaming Protocol) est un protocole de
		communication destiné aux systèmes de streaming. Il permet de contrôler un serveur de média à distance, offrant des fonctionnalités
		classiques d'un lecteur vidéo telles que lecture et pause.
		Exemple de requête:
		\begin{lstlisting}
		PLAY rtsp://myserver/axis-media/media.amp?videocodec=h264&resolution=640x480
		RTSP/1.0 CSeq: 4
		User-Agent: Axis AMC
		Session: 12345678
		Authorization: Basic cm9vdDpwYXNz
		\end{lstlisting}
		Cette requête permet de mettre en marche la lecture de la caméra, ou de
		reprendre la lecture en cas de mise en pause.
		\newline
