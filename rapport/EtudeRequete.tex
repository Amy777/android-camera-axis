	\section{Les protocoles de transfert}
		\subsection{HTTP}
		Les requétes HTTP permettent d'intéragir de multiples façon avec la caméra.
		Celles-ci permettent d'utiliser le mécanisme PTZ de la caméra (Pan Tilt
		Zoom), de faire une capture d'écran, d'activer la détection de mouvements,
		etc. 
		En effet, la méthode GET transmet les données via l'url, ce qui permet
		d'utiliser toutes les fonctionnalités de la caméra. 
		Exemple d'url:
		\begin{lstlisting}
		http://<nom-serveur>/axis-cgi/jpg/image.cgi?resolution=320*240&camera=1
		\end{lstlisting}
		Permet de récuperer un capture au format jpg, de résolution 320*240 pixels.
		\begin{lstlisting}
		http://<nom-serveur>/axis-cgi/ptz/ptzupdate.cgi?pan=15&tilt=25
		\end{lstlisting}
		Fait bouger la caméra de 15 unités vers la droite et de 25 vers le haut.
		
		Si la requette necessite de recevoir des données, celle-ci seront contenue
		dans la reponse. Par exemple, dans le cas où nous envoyons la requete 
		permettant de récupéré une capture (decrite ci-dessus). L'image sera contenu
		dans la reponse sous la forme : 
		\begin{lstlisting}
HTTP/1.0 200 OK\r\n
Content-Type: image/jpeg\r\n
Content-Length: 15656\r\n
\r\n
<JPEG image data>\r\n
\end{lstlisting}
		\subsection{RTSP}
		Real Time Streaming Protocol (protocole de streaming en temps-réel) est un
		protocole de communication destiné aux systémes de streaming. Il permet
		de contréler un serveur de média à distance, offrant des fonctionnalités
		typiques d'un lecteur vidéo telles que lecture et pause.
		Exemple de requéte:
		\begin{lstlisting}
		PLAY rtsp://myserver/axis-media/media.amp?videocodec=h264&resolution=640x480
		RTSP/1.0 CSeq: 4
		User-Agent: Axis AMC
		Session: 12345678
		Authorization: Basic cm9vdDpwYXNz
		\end{lstlisting}
		Cette requéte permet de mettre en marche la lecture de la caméra, ou de
		reprendre la lecture en cas de mise en pause.
