	\section{Les formats videos}
		\subsection{MPEG-4}
		\subsubsection{Définition}
		Le MPEG4 est une norme de codage vidéo. Celui-ci
		permet de gérer toutes les nouvelles applications multimédias comme le téléchargement et le
		streaming sur Internet, le multimédia sur téléphone mobile, etc.
		\subsubsection{Avantages}
		- L'avantage principal du MPEG-4 est qu'il permet de s'adapter à beaucoup de
		supports, tels ceux cités ci-dessus.
		- Permet également de transmettre le son avec la video.
		\subsubsection{Inconvénients}
		- Utilisation impossible avec HTTP
		
		\subsection{M-JPEG ou Motion JPEG}
		\subsubsection{Définition}
		le M-JPEG est un codec vidéo qui compresse les images une à une en JPEG.
		\subsubsection{Avantages}
		- Permet une utilisation avec HTTP
		- Compression des images plus rapide que le MPEG-4
		
		\subsubsection{Inconvénients}
		- Non transmission du son
		
	\section{Les formats audio}
		\subsection{G.711}
			\subsubsection{Définition}
			Le G.711 est une norme de compression audio.
				* Échantillonnage : 8000 Hz 
				* Bande passante sur le réseau : 64 ou 56 kbit/s
				* Type de codage : MIC (Modulation d'impulsion codée)
		\subsection{G.726}
			\subsubsection{Définition}
			Le G.726 est une norme de compression audio.
				* Bande passante sur le réseau : 16, 24, 32 ou 40 kbit/s
				* Type de codage : Modulation par impulsions et codage différentiel adaptatif (MICDA)
		\subsection{AAC}
			\subsubsection{Définition}
			Le AAC (Advance Audio Coding), est un algorithme de compression audio, qui a
			pour but de réduite la qualité, pour offrir un meilleur débit binaire.
